\href{https://travis-ci.com/eengl/fortran-strings}{\texttt{ }}

\subsection*{Introduction}

fortran-\/strings is a Fortran library and module which contains functions for common string manipulations. The function ideas originate mainly from Python\textquotesingle{}s built-\/in \href{https://docs.python.org/3.7/library/stdtypes.html\#string-methods}{\texttt{ string}} functions. The functions are accessible by using {\ttfamily strings} module in your Fortran code

{\ttfamily use strings}

The function names are prefixed with {\ttfamily str\+\_\+}. The following is a list of available functions\+:


\begin{DoxyItemize}
\item {\bfseries{str\+\_\+count}} -\/ Count the occurrences of a substring in a string.
\item {\bfseries{str\+\_\+replace}} -\/ Replace a substring with another substring within a parent string.
\item {\bfseries{str\+\_\+upper}} -\/ Convert all letters to uppercase.
\item {\bfseries{str\+\_\+lower}} -\/ Convert all letters to lowercase.
\item {\bfseries{str\+\_\+split}} -\/ Split string based on a character delimiter and return string given by the column number.
\item {\bfseries{str\+\_\+uniq}} -\/ Removed duplicative entries from a delimited string.
\item {\bfseries{str\+\_\+zfill}} -\/ Pad a string with zeroes (\char`\"{}0\char`\"{}) to specified width. If width is $<$= input string width, then the original string is returned.
\item {\bfseries{str\+\_\+center}} -\/ Center a string to a specified width. The default character to fill in the centered string is a blank character.
\item {\bfseries{str\+\_\+reverse}} -\/ Reverse a string.
\end{DoxyItemize}

All functions return a deferred-\/length, allocatable character scalar ({\ttfamily character(len=\+:), allocatable}) with the exception of $\ast$$\ast${\ttfamily str\+\_\+count}$\ast$$\ast$ which returns an integer.

\subsection*{Requirements}


\begin{DoxyItemize}
\item Fortran compiler (tested with gfortran 4.\+8.\+4 and later)
\end{DoxyItemize}

\subsection*{Build and Installation}

Admittedly, my knowledge of automake, autotools, etc is not strong at this time. The makefile is preconfigured to compile with G\+NU Fortran (gfortran) {\ttfamily \$\+FC} and its appropriate compiler options {\ttfamily \$\+F\+F\+L\+A\+GS}. The default install path is set to {\ttfamily /usr/local} via {\ttfamily \$\+P\+R\+E\+F\+IX}. To change these make variables, simply edit the makefile or set these variables on the command line prior to the make commands.

\subsubsection*{Build}

{\ttfamily $>$ \mbox{[}FC=... F\+F\+L\+A\+GS=\char`\"{}...\char`\"{} P\+R\+E\+F\+IX=\char`\"{}...\char`\"{}\mbox{]} make}

\subsubsection*{Test}

Prior to {\ttfamily make install}, though optional, it is good practice to test the build with via

{\ttfamily $>$ make test}

\subsubsection*{Install}

{\ttfamily $>$ \mbox{[}sudo\mbox{]} \mbox{[}FC=... F\+F\+L\+A\+GS=\char`\"{}...\char`\"{} P\+R\+E\+F\+IX=\char`\"{}...\char`\"{}\mbox{]} make install} $\ast$$\ast$(sudo access required to install to system area)$\ast$$\ast$ 